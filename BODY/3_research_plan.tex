%%%%%%%%%%%%%%%%%%%%%%%%%%%%%%%%%%%%%%%%%%%%%%%%%%
%%%%		~~~~ Method ~~~~
%%%%%%%%%%%%%%%%%%%%%%%%%%%%%%%%%%%%%%%%%%%%%%%%%%


\chapter{Research Questions}
\label{chap:researchquestions}
\pagestyle{fancy}

In the previous section, we established the potential of generated music in serious games (section \ref{section:motivation}), and the need for adequate time-varying controls over the generation process (section \ref{section:control}) to reliably ensure the usability of the generated music in the context of MACT. We identified inner metric weight (section \ref{section:rhytmic_weight}) as a promising target to describe rhythmic structure. Research links inner metric weight to rhythmic entrainment, and perceived and observed difficulty for a player to follow/tap along to the music. Additionally, it extends prior work on controlled music generation with a powerful rhythmic feature that is interpretable, relatively concise, and calculated symbolically. 
We collected crucial technical considerations for music generation, including the overall approach in section \ref{section:non_neural_generation}, architecture in section \ref{section:deep_learning_generation}, representation in section \ref{section:symbolic_audio} and tokenization in section \ref{section:symbolic_tok}. Finally, we discussed promising methods of adding control to a model in section \ref{section:addingcontrol} and how to evaluate a model and its outputs (section \ref{section:evaluation}). 

The focus of the thesis is to develop a model that generates a complete musical piece with time-varying controls for rhythmic structure. The output of the model will be investigated for successful integration of control, player enjoyment, and interactive potential in the context of MACT. 
  
\begin{enumerate}
\item{Research Question 1}: How do we effectively control for rhythmic structure in generated music?
\begin{itemize}
\item{Sub Question 1.1}: How do we add control while benifitting from a previously trained model's capabilities?
\item{Sub Question 1.2}: How effective is our control of rhythmic structure in the resulting model?
\end{itemize}
\item{Research Question 2}: How appropriate is music generated with control of rhythmic structure for the context of MACT?
\begin{itemize}
\item{Sub Question 2.1} Do listeners reliably recognize or follow changes in rhythmic structure perceived in the music we generate?
\item{Sub Question 2.2} Are there particular differences in rhythmic structure that are easier to recognize, what are their qualities?
\item{Sub Question 2.3} Are there particular rhythmic structures that are easier to follow, what are their qualities? 
\end{itemize}
\item {Research Question 3}: Does the generated music improve player enjoyment and engagement over Chalkiadakis' \cite{Chalkiadakis_2022} rule-based system? 
\end{enumerate}
