%%%%%%%%%%%%%%%%%%%%%%%%%%%%%%%%%%%%%%%%%%%%%%%%%%
%%%%		~~~~ Method ~~~~
%%%%%%%%%%%%%%%%%%%%%%%%%%%%%%%%%%%%%%%%%%%%%%%%%%


\chapter{Research Questions}
\label{chap:researchquestions}
\pagestyle{fancy}

In the previous section, we established the potential of generated music in serious games (section \ref{section:motivation}), and the need for adequate time-varying controls over the generation process (section \ref{section:control}) to reliably ensure the usability of the generated music in the context of MACT. We identified inner metric weight \ref{section:rhytmic_weight} as a promising target feature for control, due to its established link to rhythmic entrainment, and perceived and observed difficulty for a player to follow the music. Additionally, it extends prior work on controlled music generation with a powerful rhythmic feature that is interpretable, relatively concise, and calculated symbolically. 
We collected crucial technical considerations for music generation including the overall approach in section \ref{section:non_neural_generation}, architecture in section \ref{section:deep_learning_generation}, representation in section \ref{section:symbolic_audio} and tokenization in section \ref{section:symbolic_tok}. Finally, we discussed promising methods of adding control to a model in section \ref{section:addingcontrol} and how to evaluate a model and its outputs \ref{section:evaluation}. 

The focus of the thesis is to develop a model that generates a complete musical piece with time-varying controls for inner metric weight. The output of the model will be investigated for successful integration of control, player enjoyment, and interactive potential in the context of MACT. 
  
\begin{enumerate}
\item{Research Question 1}: Can we effectively control for inner metric weight in generated music?

\item{Research Question 2}: Do shifts in metric weight provide a recognizable in-game cue?

\item {Research Question 3}: Does the generated music improve player enjoyment and engagement over Chalkiadakis\cite{Chalkiadakis_2022} rule based system. 
\end{enumerate}


