%%%%%%%%%%%%%%%%%%%%%%%%%%%%%%%%%%%%%%%%%%%%%%%%%%
%%%%		~~~~ Abstract ~~~~
%%%%%%%%%%%%%%%%%%%%%%%%%%%%%%%%%%%%%%%%%%%%%%%%%%
%
% Should be really brief, aim for 300 words, max 500

\begin{abstract}

\noindent
% THIS IS JUST AN EXAMPLE OF THE STRUCTURE
Music is essential in video games, enhancing immersion and engagement, but players may disengage due to excessive repetition or personal preferences. This is particularly problematic in games like Last Minute Gig, a Musical Attention Control Training (MACT) application for Parkinson’s patients. Repeated play, and adaptive stimuli are necessary for the training to be successful. To maintain engagement and with that the effect of the intervention, a diverse set of controlled, adaptive music is needed. Advances in generative music offer a scalable solution. This project explores efficient methods of adding control to pre-trained models, to steer towards rhythmically adaptive music suitable for MACT. Insights from small-scale experiments will inform the application of rhythmic control in RhythmLang, a transformer based music generator, which will be evaluated for interactive potential in the context of MACT.

\end{abstract}
